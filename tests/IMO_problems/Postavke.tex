\documentclass{article}

\newtheorem{z}{Exercise}

\begin{document}
    \begin{z} \textup{(1978 British Math Olympiads (BrMO))}
        An altitude of a tetrahedron is a line through a vertex perpendicular to the opposite face. Prove that the four altitudes of a tetrahedron are concurrent if and only if each edge of the tetrahedron is perpendicular to its opposite edge.
    \end{z}

    \begin{z} \textup{(1994-95 British Math Olympiads (BrMO) Round 1 P2)}
        \textup{ABCDEFGH} is a cube of side \textup{2} with \textup{A,B,C,D} above \textup{F,G,H,E}, respectively.
        \textup{M} is the midpoint of \textup{BC} and \textup{N} is the midpoint of \textup{EF}.
        Let \textup{P} be the midpoint of \textup{AB}, and \textup{Q} the midpoint of \textup{HE}. 
        Let \textup{AM} meet \textup{CP} at \textup{X}, and \textup{HN} meet \textup{FQ} at \textup{Y}. Find the length of \textup{XY}.
    \end{z}
    The solution of this exercise is $|XY| = \frac{2\sqrt{11}}{3}$. We are going to prove that this solution is true.

    \begin{z} \textup{(1984 British Math Olympiads2 (BMO2) British FIST p2)}
        \textup{ABCD} is a tetrahedron with \textup{DA = DB = DC} and \textup{AB = BC = CA}. 
        \textup{M} and \textup{N} are the midpoints of \textup{AB} and \textup{CD}. A plane $\pi$ passes through 
        \textup{MN} and cuts \textup{AD} and \textup{BC} at \textup{P} and \textup{Q} respectively.
        Prove that \textup{AP/AD = BQ/BC}.
    \end{z}

    \begin{z} \textup{(1966 International Mathematical Olimpiad Longlist (IMO ILL) Problem 17)}
        Let $ABCD$ and $A_1B_1C_1D_1$ be two arbitrary parallelograms in the space, 
        and let M, N, P, Q be points dividing the segments $AA_1$, $BB_1$, $CC_1$, $DD_1$ in equal ratios.
        Prove that the quadrilateral MNPQ is a parallelogram.
    \end{z}

    \begin{z} \textup{(1972 International Mathematical Olimpiad Shortlist (IMO ISL) Problem 5)}
        Prove the following assertion: The four altitudes of a tetrahedron ABCD intersect in a point if and only if 
        $AB^2 + CD^2 = BC^2 + AD^2 = CA^2 + BD^2$
    \end{z}

    \begin{z} \textup{(1981 International Mathematical Olimpiad Shortlist (IMO ISL) Problem 2)}
        A sphere S is tangent to the edges AB,BC,CD,DA of a tetrahedron ABCD at the points E,F,G,H respectively. The points E,F,G,H are the vertices of a square. Prove that if the sphere is tangent to the edge AC, then it is also tangent to the edge BD. 
    \end{z}

    \begin{z} \textup{(1991 International Mathematical Olimpiad Shortlist (IMO ISL) Problem 7)}
        ABCD is a terahedron: AD + BD = AC + BC, BD + CD = BA + CA, CD + AD = CB + AB;  M,N,P are the mid--points of BC, CA, AB. 
        OA = OB = OC = OD. Prove that  $\angle MOP = \angle NOP = \angle NOM$. 
    \end{z}

    \begin{z} \textup{(1966 International Mathematical Olimpiad Longlist (IMO ISL) Problem 56)}
        In a tetrahedron, all three pairs of opposite (skew) edges are mutually perpendicular. Prove that the midpoints of the six edges of the tetrahedron lie on one sphere. 
    \end{z}

    \begin{z} \textup{1979 (Austrian Polish Mathematical Competition (APMC) Team 2)}
        Let A, B, C, D be four points in space, and M and N be the midpoints of AC and BD, respectively. 
        Show that $AB^2+BC^2+CD^2+DA^2$ = $AC^2+BD^2+4MN^2$.
    \end{z}

    \begin{z} \textup{1982 (Austrian Polish Mathematical Competition (APMC) Team 2)}
        Let P be a point inside a regular tetrahedron ABCD with edge length 1. Show that 
        $d(P,AB)+d(P,AC)+d(P,AD)+d(P,BC)+d(P,BD)+d(P,CD) \ge 3/2 \sqrt{2}$ , 
        with equality only when P is the centroid of ABCD. Here d(P,XY) denotes the distance from point P to line XY.
    \end{z}

    We are only going to prove equality case.

    \begin{z} \textup{(1986 Balkan Mathematical Olympiads (BMO) Problem 2 (BUL))}
        Let E, F, G,H , K, L respectively be points on the edges AB, BC, CA, DA, DB, DC  of a tetrahedron ABCD. 
        If $AE \cdot BE = BF \cdot CF = CG\cdot AG = DH\cdot AH = DK\cdot BK = DL\cdot CL$,  prove that the points E, F, G, H, K, L lie on a sphere. 
    \end{z}

    \begin{z} \textup{(2015 Caucasus Mathematical Olympiads grade XI problem 4)}
        The midpoint of the edge SA of the triangular pyramid of SABC has equal distances  from all the vertices of the 
        pyramid. Let SH be the height of the pyramid. Prove that $BA^2 + BH^2 = CA^2 + CH^2$.
    \end{z}

    \begin{z} \textup{(2009 International Matehmatics Tournament of Towns (ToT) Fall Senior Problem 2)}
        A, B, C, D, E and F are points in space such that AB is parallel to DE, BC is parallel to EF, CD is parallel to FA, 
        but $AB \neq DE$. Prove that all six points lie in the same plane.
    \end{z}

    \begin{z} \textup{(2013 Saint Petersburg Mathematical Olympiads grade XI P3)}
        Let M and N are midpoint of edges AB and CD of the tetrahedron ABCD, AN = DM and CM = BN. Prove that AC = BD.
    \end{z}

\section*{Problems from \cite{}}
    \begin{tabular}{l|l|c|c}
        Problem in \cite{} & Source             & Proving efficiency & NDG interpretability \\
        16.1               & Russian 1985       & 0                  & 0                    \\
        16.2               & China 1978         & 0                  & 0                    \\
        16.8               & Canada 1989        & 0                  & 0                    \\
        \hline
    \end{tabular}
    
    \begin{z} \textup{16.1 (11th All-Russian Mathematics Olympiad, 1985)}
        In space, there are three line segments of equal length. Prove that there exists a plane such that the projections of these three line segments on this plane are also equal.
    \end{z}

    \begin{z} \textup{16.2 (Liaoning Province Mathematics Competition, 1978)}
        From a point $P$ outside a plane $M$, three oblique lines are drawn to the plane $M$. These three oblique lines lie in the same plane, and their feet of perpendiculars on $M$ are sequentially labeled $A$, $B$, and $C$. The angles formed between the oblique lines and the plane $M$ are denoted as $\alpha$, $\beta$, $\alpha$, respectively. Also, the distances $AB = a$, $BC = b$. Find the distance from the point $P$ to the plane $M$. \\
    \end{z}
    In this example in order  to find distance, it should be proved that $\angle QAB = \angle QCB$.

    \begin{z} \textup{16.3 (Fujian Province Mathematics Competition, 1962)}
        From a point $A$ outside a plane $M$, two oblique lines $AB$ and $AC$ are drawn towards the plane $M$, such that $\angle BAC = \alpha$. The angles between $AB$, $AC$, and the plane $M$ are $\beta$ and $\gamma$, respectively.

        \begin{enumerate}
            \item Find the angle formed by the projections of $AB$ and $AC$ onto the plane $M$.
            \item Find the angle between the plane $ABC$ and the plane $M$.
        \end{enumerate}
    \end{z}

    Nisam radila ovaj zadatak, sta ovde ima da se dokazuje? I u dokazu ne mogu da shvatim sta ima da se dokazuje.

    \begin{z} \textup{16.4 (Chinese High School Mathematics League, 1992)}
        Let $l$ and $m$ be two skew (non-coplanar) lines. On $l$, there are three points $A$, $B$, and $C$, such that $AB = BC$. Through $A$, $B$, and $C$, perpendiculars $AD$, $BE$, and $CF$ are drawn to $m$, and the feet of the perpendiculars are $D$, $E$, and $F$, respectively. Given $AD = \sqrt{3}s$ and $BE = CF = s$, find the distance between $l$ and $m$.
    \end{z}

    Nisam radila ovaj zadatak, sta ovde ima da se dokazuje? I u dokazu ne mogu da shvatim sta ima da se dokazuje.

    \begin{z} \textup{16.6 (Shanghai Mathematics Competition, 1956)}        
        There is a balloon in the sky. At point $A$, directly west of the balloon, its angle of elevation is measured as $45^\circ$. Simultaneously, at point $B$, located $45^\circ$ east of due south from the balloon, its angle of elevation is measured as $67^\circ 30'$. The distance between points $A$ and $B$ is 266 meters, and both measurement points are 1 meter above the ground. How high above the ground is the balloon at the time of measurement?
    \end{z}

    Nisam radila ovaj zadatak, sta ovde ima da se dokazuje? I u dokazu ne mogu da shvatim sta ima da se dokazuje.

    \begin{z} \textup{16.8 (Canadian Mathematical Olympiad Training, 1989)}
        In space, there are three lines $a$, $b$, and $c$. Lines $a$ and $b$ are not perpendicular to line $c$. Points $P$ and $Q$ vary on lines $a$ and $b$, respectively, such that $PQ$ is perpendicular to $c$. 

        The plane passing through point $P$, perpendicular to $b$, intersects $c$ at $R$. Similarly, the plane passing through point $Q$, perpendicular to $a$, intersects $c$ at $S$. 
    \end{z}
\end{document}

\begin{thebibliography}{10}
    @article{shao2016challenging,
    title={Challenging theorem provers with Mathematical Olympiad problems in solid geometry},
    author={Shao, Changpeng and Li, Hongbo and Huang, Lei},
    journal={Mathematics in Computer Science},
    volume={10},
    pages={75--96},
    year={2016},
    publisher={Springer}
    }

    
%     Wu, Z., Wang, L., Liu, Y.: The Dictionary of International Mathematical Olympiads, Volume of Geometry. Hebei Children Press,
% (2012). Available in Chinese: http://vdisk.weibo.com/s/A1eiYtmuHm1Vn
\end{thebibliography}